\section{Dataset}
In this work, we focused on electrons interactions inside an electromagnetic calorimeter inspired by the LHCb detector at the CERN LHC \cite{Alves:2008zz}. The calorimeter in this study uses "shashlik" technology of alternating scintillating tiles and lead plates. The prototype  consists of 5 $\times$ 5 blocks of size 12 cm $\times$ 12 cm, the cell granularity corresponds to each block being 6 $\times$ 6 of size 2 cm $\times$ 2 cm. There are 66 total layers in ECAL, 2 mm lead absorber and 4 mm scintillator each. In fact, the shower appears in 3d, but all energies deposited in all scintillator layers of one cell are summed up. This procedure reproduced the actual shower energy collection in the calorimeter. Thus, the calorimeter response can be represented as 30 $\times$ 30 images $Y$ with the corresponding parameters $(p_x,~ p_y,~ p_z,~ x,~ y)$ of the original particle. An example of such an image is presented in the top row of~\cref{fig:geant_vs_ours}.

The training data set is created as follows. The calorimeter prototype structure described above is described in \geant as a  mixture of subsequent sensitive and insensitive volumes. Particles are generated using a particle gun. Particle energies are distributed dropping as $1/E$ in the energy range between 1 and 100 GeV. Particle positions are generated uniformly in the square 1$\times$1 cm in the centre of the calorimeter face. Finally, particle angles are distributed normally with widths of 20 degrees in $XZ$ plane and 10 degrees in $YZ$ plane. Then \geant is used to simulate particle interaction with the calorimeter using the full set of corresponding physics processes. Information about every event, therefore, includes the original particle parameters accompanied by 30 $\times$ 30 matrix of energies deposited in scintillators for every cell tower $Y$. Electrons are used as test particles. Produced training dataset contains 50 000 events, and another 10 000 events are used as a  test data sample.
