

\section{Related work: GANs basics and GANs in HEP}
Generative models are of great interest in deep learning. With these models, one can approximate a very complex distribution defined as a set of samples. 
For example, such models can be utilized to generate a face image of a non-existing person or to continue a video sequence given several initial frames. 
In this section, we give a brief overview of the most popular generative model in computer vision — Generative Adversarial Networks (GANs),
 its strong and weak sides and different modifications to alleviate its weaknesses. Then, we review and analyse current approaches for applying GANs to the simulation of calorimeters in High energy physics.

